\chapter{Background} \label{sec:background}

Our work builds on a variety of prior work both by NASA in relation to the Mars
data used, and on exsiting cross-view pipelines on earth. However, due to large
differences in the available data, most existing pipelines are not a drop-in
solution.

\section{The Perseverance (M2020) Rover}

The Mars 2020 mission is the most recent rover mission to Mars, deploying the
Perseverance rover. The Perseverance rover is structurally very similar to the
more well-known Curiosity rover, of the Mars Science Labratory (MSL) mission. It
was launched for the purposes of broadening the knowledge of the Martian
surface and to explore the Jezero crater region. It also has hardware to collect
samples within the rover and in dead drops for the purposes of eventually being
returned to earth with the Mars Sample Return (MSR/SRL) mission, or with a
future mission.

The Perseverance rover makes a number of changes to the sensor suite included on
the rover. Relavent to this work, both the resolution and bit depth of the
primary navigation cameras on the mast of the rover (Navcam) were upgraded,
having a 1600x1200 pixel resolution ($\sim$2 megapixels) and being in full color
\cite{M2020_Willford2018}. This is in constrast to the Curiosity rover, which has
1024x1024 pixel resolution with only black and white \cite{MSLNavcam_Maki2012}.
This has the impact that the data processing pipeline used by this work, as well
as the machine learning portion, does not generalize between the two missions.
As such, the newer and higher-quality Perseverance rover imagery has been chosen
for usage in this work.

\section{HiRISE}



\section{Diffusion Models}

\section{ControlNet}
